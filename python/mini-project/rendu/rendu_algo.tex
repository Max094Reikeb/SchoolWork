%! Author = Maxence BEKIER
%! Date = 29/10/2022

% Preamble
\documentclass[11pt]{article}

% Packages
\usepackage[document]{ragged2e}
\usepackage{algo}
\usepackage[utf8]{inputenc}
\usepackage[cyr]{aeguill}
\usepackage[T1]{fontenc}
\usepackage{amsmath}
\usepackage{url}
\usepackage[francais]{babel}
\usepackage{pdfpages}
\usepackage{listings}
\inputencoding{utf8}

% Custom colors
\definecolor{mygreen}{RGB}{28,172,0}
\definecolor{mylilas}{RGB}{170,55,241}

% Titles
\mainTitle{DU Développeur niveau II}
\logoDepartment{\includegraphics[width=3.0\textwidth]{logo_departement_si}}
\secondTitle{Algorithmique et Python\\ Mini projet : puissance 4}

% Document
\begin{document}

    \thispagestyle{plain}

    \maketitle

    \vspace{1.5cm}

    \lstset{language=Python,
        backgroundcolor=\color{cyan!10},
        breaklines=true,
        morekeywords={matlab2tikz},
        keywordstyle=\color{blue},
        morekeywords=[2]{1}, keywordstyle=[2]{\color{black}},
        identifierstyle=\color{black},
        stringstyle=\color{mylilas},
        commentstyle=\color{mygreen},
        showstringspaces=false,
        numbers=left,
        numberstyle={\tiny \color{black}},
        numbersep=9pt,
        emph=[1]{for,end,break},emphstyle=[1]\color{red},
        linewidth=16cm,
        inputencoding=utf8,
        extendedchars=true,
        literate={á}{{\'a}}1  {é}{{\'e}}1  {í}{{\'i}}1 {ó}{{\'o}}1  {ú}{{\'u}}1
            {Á}{{\'A}}1  {É}{{\'E}}1  {Í}{{\'I}}1 {Ó}{{\'O}}1  {Ú}{{\'U}}1
            {à}{{\`a}}1  {è}{{\`e}}1  {ì}{{\`i}}1 {ò}{{\`o}}1  {ù}{{\`u}}1
            {À}{{\`A}}1  {È}{{\'E}}1  {Ì}{{\`I}}1 {Ò}{{\`O}}1  {Ù}{{\`U}}1
            {ä}{{\"a}}1  {ë}{{\"e}}1  {ï}{{\"i}}1 {ö}{{\"o}}1  {ü}{{\"u}}1
            {Ä}{{\"A}}1  {Ë}{{\"E}}1  {Ï}{{\"I}}1 {Ö}{{\"O}}1  {Ü}{{\"U}}1
            {â}{{\^a}}1  {ê}{{\^e}}1  {î}{{\^i}}1 {ô}{{\^o}}1  {û}{{\^u}}1
            {Â}{{\^A}}1  {Ê}{{\^E}}1  {Î}{{\^I}}1 {Ô}{{\^O}}1  {Û}{{\^U}}1
            {œ}{{\oe}}1  {Œ}{{\OE}}1  {æ}{{\ae}}1 {Æ}{{\AE}}1  {ß}{{\ss}}1
            {ç}{{\c c}}1 {Ç}{{\c C}}1 {ø}{{\o}}1  {Ø}{{\O}}1   {å}{{\r a}}1
            {Å}{{\r A}}1 {ã}{{\~a}}1  {õ}{{\~o}}1 {Ã}{{\~A}}1  {Õ}{{\~O}}1
            {ñ}{{\~n}}1  {Ñ}{{\~N}}1  {¿}{{?`}}1  {¡}{{!`}}1
            {°}{{\textdegree}}1 {º}{{\textordmasculine}}1 {ª}{{\textordfeminine}}1,
    }

    \textbf{grille\_init(): fontion qui renvoie un tableau de 6 lignes et 7 colonnes remplies de zéros}
    \lstinputlisting{codes/exo1.txt}

    \newpage

    \textbf{affiche\_grille(tableau): fonction qui affiche la grille du jeu dans la console de la facon la plus esthétique possible}
    \lstinputlisting{codes/exo2.txt}

    \newpage

    \textbf{colonne\_libre(tableau, colonne): fonction qui renvoie un booléen indiquant s’il est possible de mettre un jeton dans la colonne}
    \lstinputlisting{codes/exo3.txt}

    \vspace{1.0cm}

    \textbf{place\_jeton(tableau, colonne, joueur): fonction qui place un jeton du joueur (1 ou 2) dans la colonne. Elle renvoie la grille modifiée}
    \lstinputlisting{codes/exo4.txt}

    \newpage

    \textbf{horizontale(tab, joueur): fonction qui renvoie True si le joueur a au moins 4 jetons alignés dans une ligne}
    \lstinputlisting{codes/exo5.txt}

    \vspace{1.0cm}

    \textbf{verticale(tableau, joueur): fonction qui renvoie True si le joueur a au moins 4 jetons alignés dans une colonne}
    \lstinputlisting{codes/exo6.txt}

    \newpage

    \textbf{diagonale(tableau, joueur): fonction qui renvoie True si le joueur a au moins 4 jetons alignés dans une diagonale}
    \lstinputlisting{codes/exo7.txt}

    \vspace{1.0cm}

    \textbf{gagne(tableau, joueur): fonction qui renvoie True si le joueur a gagné}
    \lstinputlisting{codes/exo8.txt}

    \newpage

    \textbf{egalite(tableau): fonction qui renvoie True s’il y a égalité et False sinon}
    \lstinputlisting{codes/exo9.txt}

    \newpage

    \textbf{tour\_joueur(tableau, joueur): fonction qui permet au joueur de placer un jeton dans la colonne choisie. Elle indique si la colonne est pleine et permet alors au joueur de choisir une autre colonne}
    \lstinputlisting{codes/exo10.txt}

    \newpage

    \textbf{jouer(tableau): fonction qui permet aux deux joueurs de jouer chacun leur tour. Elle vérifie que les joueurs n’ont pas gagné à la fin de leur tour. Si l’un des deux a gagné ou s’il y a égalité, elle donne le résultat.}
    \lstinputlisting{codes/exo11.txt}

    \newpage

    \textbf{Programme principal}
    \lstinputlisting{codes/exo12.txt}

\end{document}
